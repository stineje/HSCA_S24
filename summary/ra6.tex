\documentclass{article}
\usepackage{url}
\usepackage{fancyhdr}
\usepackage{amsmath, amsfonts, amsthm, amssymb}  
\usepackage{secdot}
\usepackage{epsfig}
\usepackage{lastpage}
\usepackage{hyperref}
\usepackage{graphicx}
\usepackage{color}
\usepackage{epstopdf}
\usepackage{fancyvrb}

\usepackage{geometry}
\geometry{letterpaper, left=1in, right=1in, top=1in, bottom=1in}

\pagestyle{fancy}
\fancyhf{}
\renewcommand{\headrulewidth}{0pt}
\rfoot{\thepage/\pageref{LastPage}}
\lhead{OSU ECEN 4233 - High-Speed Computer Arithmetic - Spring 2021}
\lfoot{\LaTeX}

\begin{document}

\title{Reduced Area Multiplier Example}
\author{James E. Stine \\
Electrical and Computer Engineering Department\\
Oklahoma State University \\
Stillwater, OK 74078, USA}
\date{}

\maketitle
%\thispagestyle{plain}\pagestyle{plain}

\section{Reduced Area Multiplier}
This document is meant to show the in-class example of a $6$-bit by $6$-bit
Reduced-Area or Column-Compression Multiplier.  
Here is
a table documenting the area where the final iteration has an $7$-bit CPA:
  \begin{table} [h]
    \centering
    \begin{tabular}{|c|c|c|c|c|c|c|c|c|c|} \hline
    Iteration & Number of $(3,2)$ Counters & Number of $(2,2)$ Counters \\
    \hline \hline
      1 & 8 & 2  \\ \hline
      2 & 6 & 1  \\ \hline
      3 & 4 & 2  \\ \hline \hline
  Total & 18 & 5 \\ \hline
    \end{tabular}
  \end{table}

The methodology
for creating Reduced-Area trees, or so they are called, can
be organized into the following steps listed below.  
\begin{enumerate}
\item Reorganize matrix into inverted triangle (optional)
\item For each stage, the number of Full Adders (FAs) used in the column $i$ is
  $\#FAs = \lfloor b_i/3 \rfloor$ where $b_i$ is the number of bits in column
  $i$.
\item Half Adders (HAs) are used only if
\begin{enumerate}
 \item when required to reduce the number of bits in a column to the height
  specified by the Dadda sequence.
 \item To reduce the rightmost column containing exactly two bits.
\end{enumerate}
\item Repeat step $(2)$ until the final height is $2$.
\end{enumerate}

  \begin{figure}
    \begin{center}
      \setlength{\unitlength}{0.0105in}%
      \epsfig{figure=ra6.eps, height=5.0in}
    \end{center}
    \label{wallace6.fig}
    \caption{In-class Example of $6 \times 6$ Reduced-Area or Column-Compression Multiplier.}
  \end{figure}

\end{document}
