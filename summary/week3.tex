\documentclass[times, twocolumn, 10pt]{article}
\usepackage{url}
\usepackage{fancyhdr}
\usepackage{amsmath, amsfonts, amsthm, amssymb}  
\usepackage{epsfig}
\usepackage{lastpage}
\usepackage{hyperref}
\usepackage{graphicx}
\usepackage{color}
\usepackage{epstopdf}
\usepackage{fancyvrb}

\usepackage{geometry}
\geometry{letterpaper, left=1in, right=1in, top=1in, bottom=1in}

\pagestyle{fancy}
\fancyhf{}
\renewcommand{\headrulewidth}{0pt}
\rfoot{\thepage/\pageref{LastPage}}
%\lhead{OSU ECEN 4233 - High-Speed Computer Arithmetic - Spring 2021}
\lfoot{\LaTeX}

\begin{document}

\title{Carry-Skip and Carry-Select Adders}
\author{James E. Stine \\
Electrical and Computer Engineering Department\\
Oklahoma State University \\
Stillwater, OK 74078, USA}
\date{}

\maketitle
%\thispagestyle{plain}\pagestyle{plain}

\begin{enumerate}
\item Carry Skip Concept. 
  \begin{enumerate}
  \item The carry skip adder (CSKA) divides the operands to be added into $r$ bit 
    blocks, the average size of which is $n/r$. Within each block, a ripple 
    carry adder is used to produce the sum bits and a carry out
    bit for the block. 
  \item Setting the carry-in signal of a block to zero causes the carry out 
    to serve as a block generate signal. 
  \item An $r$ bit AND gate is also used to form the block propagate signal. 
  \item The block generate and block propagate signals are combined using
    the standard carry equation to produce the input carry to the next 
    block. 
  \end{enumerate}
\item A 16-bit CSKA adder with 4-bit blocks. 
  \begin{enumerate}
  \item Draw picture of the 16-bit CSKA adder with 4-bit blocks. 
  \item In order to obtain the carry into bit position 8, we use the 
    equation:
    \begin{eqnarray*}
      c_{8} = g_{7:4} + p_{7:4} \cdot c_{4} 
    \end{eqnarray*}
    where 
    \begin{eqnarray*}
      p_{7:4} = p_{7} \cdot p_{6} \cdot p_{5} \cdot p_{4}
    \end{eqnarray*}
  \item The first and the fourth blocks are regular ripple carry adders, 
    while the second and the third blocks are ripple carry adders, 
    with three additional gates. 
  \item This adder requires $16 \cdot 9 = 144$ gates to implement the FAs
    and $2 \cdot 3 = 6$ gates to implement the carry logic, for a total 
    of $150$ gates. 
  \item It is assumed that the carry logic would be $2$ gates, even though
  this would require a $5$-input AND gate (e.g. $p_7 \cdot p_6 \cdot p_5
  \cdot p_4 \cdot c_4$).  The equations could easily be changed to handle
  only $4$-input gates or less, but this would increase the gate count.
  \item The delay for this adder is $2 \cdot 4 + 3 = 11 \bigtriangleup$ to 
    go through the first ripple carry adder $2 \cdot 2 = 4 \bigtriangleup$ to
    go through the next two blocks, and $2 \cdot 4 + 1 = 9 \bigtriangleup$ 
    to go through the next last block, for a total delay of $24
    \bigtriangleup$. 
  \end{enumerate}
\item Generalized  CSKA Gate Counts
  \begin{enumerate}
  \item An $n$-bit CSKA uses $n$ FAs, each of which requires $9$ gates. 
  \item It also uses $\lceil n/r \rceil - 2$ sets of carry skip logic, each 
    of which requires $3$ gates. 
  \item Thus, the total number of gates used by an $n$-bit CSKA is:
    \begin{displaymath}
      9 \cdot n + 2 \cdot (\lceil \frac{n}{r} \rceil - 2)
    \end{displaymath}
  \end{enumerate}
\item Generalized  CSKA Delay
  \begin{enumerate}
  \item The first block has a delay of $(2 \cdot r + 3) \bigtriangleup$
  before the 
    carry out is ready. 
  \item The next $(\lceil n/r \rceil - 2)$ blocks have a delay of 
    $2 \bigtriangleup$  for the carry to skip. 
  \item The last block has a delay of $2 \cdot r + 1$ from the carry in to the 
    most significant sum bit. 
  \item Thus, the total delay is for $s_{n-1}$ is:
    \begin{eqnarray*}
      & & (2 \cdot r + 3)  + 2 \cdot (\lceil \frac{n}{r} \rceil - 2) + (2 \cdot
      r + 1) \\
      & = & 4 \cdot r + 2 \cdot \lceil \frac{n}{r} \rceil \\
      & \approx & 4 \cdot r + 2 \cdot \frac{n}{r}
    \end{eqnarray*}
  \end{enumerate}
\item Optimizing the Block Size to Reduce Delay
  \begin{enumerate}
  \item The optimum block size is determined by taking the derivative
    of the delay with respect to $r$, setting it to zero, and 
    solving for $r$. 
    \begin{eqnarray*}
      4 - \frac{2 \cdot n}{r^{2}} & = & 0  \\
      r & = & \sqrt{n/2} 
    \end{eqnarray*}
  \item Plugging this into the delay equation gives 
    \begin{displaymath}
      4 \sqrt{n/2} + 2 \cdot \frac{n}{\sqrt{n/2}}  = 4 \sqrt{2 \cdot n}
    \end{displaymath}
  \item For example, if $n = 32$, then the delay is minimized by
    selecting $r = 4$, which gives a worst case delay of 
    $4 \sqrt{2 \cdot 32} = 32 \bigtriangleup$. 
  \item For $n = 16$, the delay is minimized by selecting 
    $r = \sqrt{8} \approx 2.82$. Since r must be an integer, 
    we can select $r = 3$ for the first five blocks, and use
    a full adder for the remaining bit. This adder has a worst
    case delay of $22 \bigtriangleup$ and requires 156 gates. 
  \end{enumerate}
\item Variable Length Blocks. 
  \begin{enumerate}
  \item The delay of the carry skip adder can be reduced even
    further by varying the block size. 
  \item A good strategy is to use smaller blocks on the two ends
    and larger blocks in the middle. 
  \item The design of a 16 bit CSKA with block size of $(1, 2, 3, 4, 3, 2, 1)$ 
    requires 159 gates and has a worst case delay of $19 \bigtriangleup$.
  \item Speed can also be improved by using faster block adders 
    (e.g., CLAs) on the ends or using more than one level 
    of carry skip logic.
  \end{enumerate}
\item Characteristics of CSKAs
  \begin{enumerate}
  \item In general, CSKAs require slightly more gates than RCAs and fewer
    gates than CLAs. 
  \item The delay for CSKAs is less than the delay for RCAs, but 
    greater than the delay for CLAs. 
  \item CSKAs are less regular than RCAs and more regular than CLAs. 

    \begin{figure*}
      \begin{center}
	\setlength{\unitlength}{0.0105in}%
	\epsfig{figure=c4skip16.eps, height=2.0in}
      \end{center}
      \label{c4skip16.fig}
      \caption{16-bit Carry Skip Adder (r = 4).}
    \end{figure*}

    \begin{figure*}
      \begin{center}
	\setlength{\unitlength}{0.0105in}%
	\epsfig{figure=c3skip16.eps, height=1.0in}
      \end{center}
      \label{c3skip16.fig}
      \caption{16-bit Carry Skip Adder (r = 3).}
    \end{figure*}

    \begin{figure*}
      \begin{center}
	\setlength{\unitlength}{0.0105in}%
	\epsfig{figure=v2skip16.eps, height=1.0in}
      \end{center}
      \label{v2skip16.fig}
      \caption{16-bit Carry Skip Adder with Variable-Size Blocks.}
    \end{figure*}
  \end{enumerate}

\item Carry Select Concept. 
  \begin{enumerate}
  \item The carry select adder (CSEA) divides the operands to be added into
  $r$ bit 
    blocks, the average size of which is $n/r$. 
  \item For each block, except the first, two $r$-bit ripple carry adders
  operate
    in parallel to form $2$ sets of sum bits and carry out signals. One RCA 
    has an initial carry in of zero and the other has an initial carry 
    in of one.    
  \item The block with a carry in of zero provides a block generate signal, and
    the block with a carry in of one provides a block propagate signal. These
    two signals are used to generate a carry out signal for the block. 
  \item The carry out from the previous block controls a multiplexor that
  selects
    the appropriate set of sum bits. 
  \end{enumerate}
\item Multiplexor Logic. 
  \begin{enumerate}
  \item A $2:1$ multiplex, with inputs $x$ and $y$, select bit $t$, and output
  $z$, can be implemented as
    \[	z = x \cdot t + y \cdot \overline{t} \]
    If $t = 1$, $z = x$; otherwise $z = y$. 
  \item An $r$-bit $2:1$ multiplex requires $3 \cdot r + 1$ gates; $1$ to
  invert $t$ and $3 \cdot r$ 
    to implement the AND and OR gates. 
  \end{enumerate}
\item A $16$-bit CSEA adder with $4$-bit blocks. 
  \begin{enumerate}
  \item Draw picture of the $16$-bit CSEA adder with $4$-bit blocks. 
  \item In order to obtain the carry into bit position $8$, we use the 
    equation:
    \begin{eqnarray*}
      c_{8} = g_{7:4} + p_{7:4} \cdot c_{4} 
    \end{eqnarray*}
    where $g_{7:4}$, and $p_{7:4}$ come from the ripple carry adders. 
  \item The delay for this adder is $2 \cdot 4 + 3 = 11 \bigtriangleup$ to 
    go through the first ripple carry adder, $2 \cdot 2 = 4 \bigtriangleup$
    to go 
    through the next two blocks, and $3 \bigtriangleup$ to go through the
    multiplexor. The total delay is $18 \bigtriangleup$. 
  \item The adder requires $28 \cdot 9 = 252$ gates for full adders, $12
  \cdot 3 + 3 = 39$
    gates for the multiplexors, and $2 \times 3 = 6$ gates for the carry
  logic.  The total gate count is $297$ gates. 
  \end{enumerate}
\item Generalized  CSEA Gate Counts
  \begin{enumerate}
  \item An $n$-bit CSEA with $r$ bit blocks uses $2 \cdot n - r$ FAs, each of
  which requires $9$ gates. 
  \item It uses $\lceil n/r \rceil - 1$ sets of carry logic blocks, each 
    of which requires 2 gates. 
  \item The mux logic requires $\lceil n/r \rceil - 1$ inverters and 
    $3 \cdot (n - r)$ AND/OR gates. 
  \item Thus, the total number of gates used by an $n$-bit CSEA is:
    \begin{eqnarray*}
      & & 9 \cdot (2 \cdot n - r) + 2 \cdot (\lceil \frac{n}{r} \rceil - 1) + \\
      & & (\lceil \frac{n}{r} \rceil - 1) + 3 \cdot (n - r) \\ 
      & = & 21 \cdot n - 12 \cdot r + 3 \cdot \lceil \frac{n}{r} \rceil - 3 
    \end{eqnarray*}
  \end{enumerate}
\item Generalized  CSEA Delay
  \begin{enumerate}
  \item The first block has a delay of $(2 \cdot r + 3) \bigtriangleup$
  before the carry out is ready. 
  \item The next $(\lceil n/r \rceil - 2)$ blocks have a delay of 
    $2 \bigtriangleup$  for the carry logic. 
  \item The last block has a delay of 3 for the multiplexor selection logic.
  \item Thus, the total delay is for $s_{n-1}$ is:
    \begin{eqnarray*}
      2 \cdot r + 3  + 2 \cdot (\lceil \frac{n}{r} \rceil - 2) + 3 \
      = \\
      2 \cdot r + 2 \cdot \lceil \frac{n}{r} + 1 \rceil \approx \
      2 \cdot r + 2 \cdot \frac{n}{r} + 2
    \end{eqnarray*}
  \end{enumerate}
\item Optimizing the Block Size to Reduce Delay
  \begin{enumerate}
  \item The optimum block size is determined by taking the derivative
    of the delay with respect to $r$, setting it to zero, and 
    solving for $r$. 
    \begin{eqnarray*}
      2 - \frac{2 n}{r^{2}} & = & 0  \\
      r & = & \sqrt{n} 
    \end{eqnarray*}
  \item Plugging this into the delay equation gives 
    \begin{displaymath}
      2 \sqrt{n/2} + 2 \cdot \frac{n}{\sqrt{n}} + 2 = 2 \sqrt{n} + 2
    \end{displaymath}
  \item For example, if $n = 16$, then the delay is minimized by
    selecting $r = 4$, which gives a worst case delay of 
    $4 \sqrt{16} + 2 = 18 \bigtriangleup$. 
  \end{enumerate}
\item Variable Length Blocks. 
  \begin{enumerate}
  \item The delay of the CSEA can be reduced even further by varying the
  block size. 
  \item A good strategy is have the first two blocks be small and have the
  same size. Each subsequent block should have one bit larger than the
  previous block. 
  \item For example, a $16$ bit CSEA with block size of $(2, 2, 3, 4, 5)$ 
    requires $322$ gates and has a worst case delay of $16 \bigtriangleup$.
  \item Speed can also be improved by using faster block adders (e.g., CLAs) 
    and another CLA to generate the carry bits.  
  \end{enumerate}
\item Characteristics of CSEAs
  \begin{enumerate}
  \item In general, CSEAs require even more gates than CLAs. 
  \item The delay for CSEAs proportional to $\sqrt{n}$. It is less
    than the delay for RCAs and CSKAs, but greater than the delay for CLAs. 
  \item CSKAs are less regular than RCAs and CSKAs, and slightly more regular 
    than CLAs. 

    \begin{figure*}
      \begin{center}
	\setlength{\unitlength}{0.0105in}%
	\epsfig{figure=sel16.eps, height=2.0in}
      \end{center}
      \label{csel16.fig}
      \caption{16-bit Carry Select Adder (r = 4).}
    \end{figure*}

  \end{enumerate}
\end{enumerate}




\end{document}
